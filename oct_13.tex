\documentclass[12pt,letterpaper]{hmcpset}
\usepackage[margin=1in]{geometry}
\usepackage{graphicx}

% info for header block in upper right hand corner
\name{ }
\class{Math 55 - Section ~~~}
\assignment{Homework 5}
\duedate{Thursday, October 13, 2016}

\renewcommand{\t}[1]{\text{#1}}
\renewcommand{\labelenumi}{{\alph{enumi})}}

\begin{document}

\problemlist{1, 2, 3, 4, 5, 6, 7, 8}

\begin{problem}[1]
    Prove the I.C. theorem from class: If $d|a$ and $d|b$ then $d|ax+by$ for all integers $x$ and $y$.
\end{problem}
\begin{solution}
    \vfill
\end{solution}
\newpage

\begin{problem}[2]
    Prove that $n$ can be expressed as an integer combination of $a$ and $b$ if and only if $n$ is a multiple of $(a,b)$.  (Recall that an if and only if proof requires proving two directions.)
\end{problem}
\begin{solution}
    \vfill
\end{solution}
\newpage

\begin{problem}[3]
    \begin{enumerate}
        \item Without using prime factorizations, prove that for integers $a,b,m$: $(a,m)=1$ and $(b,m)=1$ if and only if $(ab,m)=1$.
        \item Do the same problem with unique factorization.
    \end{enumerate}
\end{problem}
\begin{solution}
    \vfill
\end{solution}
\newpage

\begin{problem}[4]
    Use the Euclidean algorithm to find:
    \begin{enumerate}
        \item gcd(754, 221),
        \item an integer combination of 754 and 221 yielding the gcd.
        \item integers $x$ and $y$ satisfying $87x+28y=200$.
        \item another solution to c) where the signs of $x$ and $y$ are reversed.
    \end{enumerate}
    (Note: All of your answers must utilize Euclid's algorithm. No credit for brut forcing or lucky guessing.)
\end{problem}
\begin{solution}
    \vfill
\end{solution}
\newpage

\begin{problem}[5]
    Use Euclid's algorithm to find an integer $x$ such that $97x$ is one more than a multiple of $1234$.  (This is called the \textit{multiplicative inverse} of $97,~\t{mod}~1234$.)\\\\
    (Hint: 97 and 1234 are relatively prime.)
\end{problem}
\begin{solution}
    \vfill
\end{solution}
\newpage

\begin{problem}[6]
    For Fibonacci numbers $F_n$ (with $F_0=0$ and $F_1=1$),
    \begin{enumerate}
        \item Prove that $(F_{n+1},F_n)=1$ for $n\geq0$.
        \item Prove for all $n\geq1,F_{m+n}=F_{m+1}F_n+F_mF_{n-1}.$
    \end{enumerate}
    (Hint: You can either give a combinatorial argument (recall $F_k=f_{k-1}$) or by fixing $m$ and inducting on $n$.)
\end{problem}
\begin{solution}
    \vfill
\end{solution}
\newpage

\begin{problem}[7]
    Prove that if $m|n$ then $F_m|F_n$. (Hint: Prove $F_{km}$ is a multiple of $F_m$ by inducting on $k$ and using part b) of Problem 6).\\\\
    (In fact, Problems 6a and 7 are special cases of a more amazing theorem which says that if $(m,n)=g$, then $(F_m,F_n)=F_g$. For the proof of this, take Math 175!)
\end{problem}
\begin{solution}
    \vfill
\end{solution}
\newpage

\begin{problem}[8]
    Prove the theorem ``If $(b,c)=1$, then $(ab,c)=(a,c)$'' in two different ways:
    \begin{enumerate}
        \item by showing that any common divisor of one pair is a common divisor of the other, and
        \item by showing that any integer combination of one pair is an integer combination of the other.
    \end{enumerate}
\end{problem}
\begin{solution}
    \vfill
\end{solution}
\end{document}
