\documentclass[12pt,letterpaper]{hmcpset}
\usepackage[margin=1in]{geometry}
\usepackage{graphicx}

% info for header block in upper right hand corner
\name{ }
\class{Math 55 - Section ~~~}
\assignment{Homework 3}
\duedate{Thursday, September 22, 2016}

\newcommand{\pn}[1]{\left(#1\right)}
\newcommand{\mc}[2]{\pn{\binom{#1}{#2}}}
\renewcommand{\b}[2]{\binom{#1}{#2}}
\renewcommand{\labelenumi}{{(\alph{enumi})}}

\begin{document}

\problemlist{1, 2, 3, 4, 5, 6, 7, 8, 9, 10}

\begin{problem}[1]
    Prove the identity: For $m\geq0$ and $n\geq1$,
    \[
        \sum_{k=0}^{m}(-1)^k\b{n}{k}=(-1)^m\b{n-1}{m}
    \]
    by induction on $m$. (Note: It can also be proved by induction on $n$, but induction on $m$ is simpler.)
\end{problem}
\begin{solution}
    \vfill
\end{solution}
\newpage

\begin{problem}[2]
    Combinatorially prove for $m\geq0$, $n\geq0$,
    \[
        \sum_{k=0}^n\b{n}{k}\b{k}{m}=\b{n}{m}2^{n-m}.
    \]
\end{problem}
\begin{solution}
    \vfill
\end{solution}
\newpage

\begin{problem}[3]
    Prove that for $n\geq0$,
    \[
        1+\b{n}{1}2+\b{n}{2}4+\dots+\b{n}{n-1}2^{n-1}+\b{n}{n}2^n=3^n
    \]
    \begin{enumerate}
        \item combinatorially, and
        \item by binomial theorem.
    \end{enumerate}
\end{problem}
\begin{solution}
    \vfill
\end{solution}
\newpage

\begin{problem}[4]
    For $n\geq1$, conjecture and prove a formula for the $n$ odd-indexed Fibonacci numbers $f_1+f_3+f_5+\cdots+f_{2n-1}$
    \begin{enumerate}
        \item by induction, and
        \item combinatorially (using tilings)
    \end{enumerate}
    (Note: $f_1=1,f_3=3,f_5=8$, etc.)
\end{problem}
\begin{solution}
    \vfill
\end{solution}
\newpage

\begin{problem}[5]
    Use the binomial theorem with Binet's formula to prove that for $n\geq0$,
    \[
        \sum_{k=0}^n\b{n}{k}F_k=F_{2n}
    \]
    Optional BONUS for problem 5 (turn in this solution to me, and do not work on this problem with others): Prove problem 5 combinatorially.
\end{problem}
\begin{solution}
    \vfill
\end{solution}
\newpage

\begin{problem}[6]
    Use strong induction to prove that every integer $n$ greater than or equal to 2 is the product of prime numbers. (A prime is a number $p$ greater than or equal to 2 whose only positive divisors are $p$ and 1.)
\end{problem}
\begin{solution}
    \vfill
\end{solution}
\newpage

\begin{problem}[7]
    Prove for $n\geq2$,
    \[
        \sum_{k=0}^nk^2\b{n}{k}=n2^{n-1}+n(n-1)2^{n-2}
    \]
    \begin{enumerate}
        \item combinatorially and
        \item using the binomial theorem.
    \end{enumerate}
\end{problem}
\begin{solution}
    \vfill
\end{solution}
\newpage

\begin{problem}[8]
    \begin{enumerate}
        \item How many $n$-digit positive numbers have their digits in strictly increasing order?
        \item in Non-decreasing order?
    \end{enumerate}
     (Clarification: an $n$-digit positive number cannot have a leading digit of zero. In base 4, there are 10 non-decreasing 3-digit numbers: 111, 112, 113, 122, 123, 133, 222, 223, 233, and 333.)
\end{problem}
\begin{solution}
    \vfill
\end{solution}
\newpage

\begin{problem}[9]
    \begin{enumerate}
        \item How many ways are there to pick a collection of 10 balls from a large pile of red balls, blue balls and purple balls, (more than 10 of each color, and balls of the same color are indistinguishable).
        \item same problem but you must pick at least 5 red balls.
        \item same, but instead must pick at most 5 red balls.
    \end{enumerate}
\end{problem}
\begin{solution}
    \vfill
\end{solution}
\newpage

\begin{problem}[10]
    Combinatorially prove that
    \[
        \sum_{k=0}^m\mc{n}{k}=\mc{n+1}{m}.
    \]
\end{problem}
\begin{solution}
    \vfill
\end{solution}
\end{document}
