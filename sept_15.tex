\documentclass[12pt,letterpaper]{hmcpset}
\usepackage[margin=1in]{geometry}
\usepackage{graphicx}

% info for header block in upper right hand corner
\name{ }
\class{Math 55 - Section ~~~}
\assignment{Homework 2}
\duedate{Thursday, September 15, 2016}

\newcommand{\f}[2]{\frac{#1}{#2}}
\renewcommand{\labelenumi}{{(\alph{enumi})}}

\begin{document}

\problemlist{1, 2, 3, 4, 5, 6, 7, 8}

\begin{problem}[1]
    How many ways can 12 distinct candy bars be distributed to 4 distinct children such that child 1 gets 2 bars, child 2 gets 2 bars, child 3 gets 4 bars, and child 4 gets 4 bars?
\end{problem}
\begin{solution}
    \vfill
\end{solution}
\newpage

\begin{problem}[2]
    Prove by induction:
    \[
        1+x+x^2+\dots+x^n=\f{1-x^{n+1}}{1-x},
    \]
    where $x\neq1$ and $n$ is a nonnegative integer.\\\\
    (Aside: Here's a clever non-inductive proof for your enjoyment. If we denote the sum on the left by $S$, then $Sx=x+x^2+\dots+x^n+x^{n+1}$. Thus $S(1-x)=S-Sx=1-x^{n+1}$.)
\end{problem}
\begin{solution}
    \vfill
\end{solution}
\newpage

\begin{problem}[3]
    Prove by induction that for $n\geq1$,
    \[
        1\cdot2^0+2\cdot2^1+3\cdot2^2+\dots+n\cdot2^{n-1}=(n-1)2^n+1.
    \]
\end{problem}
\begin{solution}
    \vfill
\end{solution}
\newpage

\begin{problem}[4]
    Prove by induction that for all integers $n\geq0$,
    \[
        \int_{0}^{\infty}x^ne^{-x}dx=n!
    \]
\end{problem}
\begin{solution}
    \vfill
\end{solution}
\newpage

\begin{problem}[5]
    Prove that
    \[
        \f{1}{1\cdot2}+\f{1}{2\cdot3}+\dots+\f{1}{2000\cdot2001}=\f{2000}{2001}
    \]
    \begin{enumerate}
        \item using induction, and
        \item without using induction. (Hint: Partial Fractions.)
    \end{enumerate}
\end{problem}
\begin{solution}
    \vfill
\end{solution}
\newpage

\begin{problem}[6]
    Prove that for $n\geq1$, 
    \[
        \sum_{k=0}^{n}k\binom{n}{k}=n2^{n-1}
    \]
    \begin{enumerate}
        \item by a combinatorial proof, and
        \item  by induction. Hint: Use Pascal's identity, and at some point in the proof, you might split $k$ as $(k -1)+1$. (Moral: Combinatorial proofs are more fun!)
    \end{enumerate}
\end{problem}
\begin{solution}
    \vfill
\end{solution}
\newpage

\begin{problem}[7]
    Give a proof by induction AND a non-inductive proof for both of the
following problems. For $n\geq1$,
    \begin{enumerate}
        \item $n^3-n$ is a multiple of 6.
        \item If $n$ is odd, then $n^2-1$ is a multiple of 4.
    \end{enumerate}
\end{problem}
\begin{solution}
    \vfill
\end{solution}
\newpage

\begin{problem}[8]
    Suppose that $n\geq3$ children are playing on a playground, each holding a custard pie. Suddenly every child throws their pie at the face of the child that is closest to them. Assume that all the distances between the children are distinct so this rule is well defined. Prove, by induction, that if $n$ is odd, then at least one child has no pie thrown at her.
\end{problem}
\begin{solution}
    \vfill
\end{solution}
\end{document}
