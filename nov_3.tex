\documentclass[12pt,letterpaper]{hmcpset}
\usepackage[margin=1in]{geometry}
\usepackage{graphicx}

\name{ }
\class{Math 55 - Section ~~~}
\assignment{Homework 7}
\duedate{Thursday, November 3, 2016}

\renewcommand{\t}[1]{\text{#1}}
\renewcommand{\labelenumi}{{(\alph{enumi})}}

\begin{document}
\problemlist{1, 2, 3, 4, 5, 6, 7}

\begin{problem}[1]
    Consider the recursive algorithm described in class for exponentiation, computing 
    \[
        a^n\:(\t{mod}~m),
    \]
    where the exponent $n$ is halved, or reduced by one, at each step.
    \begin{enumerate}
        \item How many multiplications (and reductions modulo $m$) does this algorithm use when $n=15$?
        \item Find a way of computing $a^{15}\:(\t{mod}~m)$ that uses fewer multiplications (and reductions modulo $m$) than in part (a).\\\\
            (Hint: $15 =3\cdot5$.)
    \end{enumerate}
\end{problem}
\begin{solution}
    \vfill
\end{solution}
\newpage

\begin{problem}[2]
    When doing modular exponentiation modulo $m$, if $a$ is relatively prime to $m$, it is possible to divide by powers of $a$, in addition to multiplying them.
    \begin{enumerate}
        \item How many multiplications (and reductions modulo $m$) does the algorithm described in class use when $n=31$?
        \item Find a way of computing $a^{31}\:(\t{mod}~m)$ that uses fewer multiplications and divisions (and reductions modulo $m$) than in part (a).
    \end{enumerate}
\end{problem}
\begin{solution}
    \vfill
\end{solution}
\newpage

\begin{problem}[3]
    Show that there are infinitely  many primes of the form $4k+3$.\\\\
    (Hint: Consider the number $N=2^2p_3\cdots p_k+3$. Note that two odd numbers of the same ``type'' mod $4$ (both $1$ or both $3$) have a product that is $1$ mod $4$, whereas two odd numbers of opposite type have a product that is $3$ mod $4$.)
\end{problem}
\begin{solution}
    \vfill
\end{solution}
\newpage

\begin{problem}[4]
    Show that $n=1729=7\cdot13\cdot19$ is a Carmichael number (that is, that even though $n$ is not prime, it satisfies
    \[
        a^{n-1}\equiv_n1
    \]
    for all $a$ relatively prime to $n$).
\end{problem}
\begin{solution}
    \vfill
\end{solution}
\newpage

\begin{problem}[5]
    How many different solutions to the congruence
    \[
        x^2\equiv1\pmod{1729}
    \]
    are there?\\\\
    (Here ``different'' means different modulo $1729=7\cdot13\cdot19$. Give a concise justification for your answer, not a brute-force search.)
\end{problem}
\begin{solution}
    \vfill
\end{solution}
\newpage

\begin{problem}[6]
    Say that a positive integer $t$ is {\it square-free\/} if it is not divisible by any square other than $1$. Show that every positive integer $n\ge1$ can be written in a unique way as $n=xy$, where $x$ is a square and $y$ is square free.
\end{problem}
\begin{solution}
    \vfill
\end{solution}
\newpage

\begin{problem}[7]
    Show that if $m_1,\ldots,m_k$ are pairwise relatively prime, then the congruences
    \[
        x\equiv_{m_1} a_1,\;\ldots,\;x\equiv_{m_k}a_k
    \]
    have a solution that is unique modulo $m_1\cdots m_k$.\\\\
    (You may use the case $k=2$, which is the Chinese remainder theorem.)
\end{problem}
\begin{solution}
    \vfill
\end{solution}
\end{document}
